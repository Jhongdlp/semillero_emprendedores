\documentclass[12pt, a4paper]{article}
\usepackage[utf8]{inputenc}
\usepackage[spanish]{babel}
\usepackage{geometry}
\usepackage{listings}
\usepackage{xcolor}
\usepackage{hyperref}

\geometry{top=2.5cm, bottom=2.5cm, left=3cm, right=3cm}

\definecolor{codegreen}{rgb}{0,0.6,0}
\definecolor{codegray}{rgb}{0.5,0.5,0.5}
\definecolor{codepurple}{rgb}{0.58,0,0.82}
\definecolor{backcolour}{rgb}{0.95,0.95,0.92}

\lstdefinestyle{mystyle}{
    backgroundcolor=\color{backcolour},   
    commentstyle=\color{codegreen},
    keywordstyle=\color{magenta},
    numberstyle=\tiny\color{codegray},
    stringstyle=\color{codepurple},
    basicstyle=\ttfamily\footnotesize,
    breakatwhitespace=false,         
    breaklines=true,                 
    captionpos=b,                    
    keepspaces=true,                 
    numbers=left,                    
    numbersep=5pt,                  
    showspaces=false,                
    showstringspaces=false,
    showtabs=false,                  
    tabsize=2
}

\lstset{style=mystyle}

\title{\textbf{Documentación Técnica del Componente de Datos} \\ \large Sistema de Gestión Integral de Semillero de Emprendedores (SGISE)}
\author{Equipo de Desarrollo}
\date{\today}

\begin{document}

\maketitle

\section{Descripción General del Proyecto}

El \textbf{Sistema de Gestión Integral de Semillero de Emprendedores (SGISE)} es una plataforma tecnológica diseñada para administrar el ciclo de vida de las startups dentro de un ecosistema de emprendimiento. 

El componente de base de datos entregado (\texttt{DB.sql}) modela la persistencia de datos crítica del sistema, abarcando:
\begin{itemize}
    \item \textbf{Gestión de Identidad y Roles:} Administración segura de emprendedores, mentores e inversores.
    \item \textbf{Ciclo de Vida de Startups:} Seguimiento desde la fase de ideación hasta el "exit" o consolidación.
    \item \textbf{Rondas de Inversión:} Registro complejo de inyecciones de capital y cálculo de participación (equity).
    \item \textbf{Seguimiento de Mentorías:} Registro de sesiones y retroalimentación cualitativa.
    \item \textbf{Histórico de KPIs:} Almacenamiento flexible de métricas de negocio mediante estructuras JSONB para análisis de datos.
\end{itemize}

\section{Acceso a la Plataforma Web}
Para complementar la validación del sistema y visualizar el flujo de entrada de datos, se ha desplegado una versión demostrativa del formulario de registro accesible públicamente:

\begin{center}
    \textbf{Formulario de Registro Web:} \\
    \href{https://semillero-emprendedores-ul1allfqg.vercel.app/formulario}{https://semillero-emprendedores-ul1allfqg.vercel.app/formulario}
\end{center}

\section{Motor de Base de Datos Utilizado}

Para este proyecto se ha seleccionado \textbf{PostgreSQL} (versión 14 o superior).

\subsection*{Justificación Técnica:}
\begin{itemize}
    \item \textbf{Soporte Avanzado de JSONB:} Permite flexibilidad en el esquema para almacenar perfiles de usuarios y métricas de startups que cambian con el tiempo, combinando lo mejor del mundo SQL y NoSQL.
    \item \textbf{Integridad Referencial Robusta:} Asegura la coherencia de los datos financieros críticos (inversiones y valoraciones).
    \item \textbf{PL/pgSQL:} Permite encapsular lógica de negocio compleja (como el cálculo de valoración post-money) directamente en la base de datos mediante procedimientos almacenados y triggers, garantizando atomicidad y rendimiento.
\end{itemize}

\section{Instrucciones de Ejecución}

El entregable consiste en un único archivo SQL llamado \texttt{DB.sql}. A continuación se detallan los pasos para su despliegue en un entorno local o servidor.

\subsection{Prerrequisitos}
\begin{itemize}
    \item Tener instalado PostgreSQL 14+.
    \item Cliente de línea de comandos \texttt{psql} o una interfaz gráfica como pgAdmin/DBeaver.
\end{itemize}

\subsection{Pasos para ejecutar (Línea de Comandos)}

1. Abra su terminal o consola de comandos.
2. Navegue hasta el directorio donde se encuentra el archivo \texttt{DB.sql}.
3. Ejecute el siguiente comando para importar el esquema y los datos simulados:

\begin{lstlisting}[language=bash]
psql -U su_usuario -d nombre_de_su_base_de_datos -f DB.sql
\end{lstlisting}

\textit{Nota: El script incluye instrucciones \texttt{DROP SCHEMA} para limpiar ejecuciones previas, por lo que es seguro ejecutarlo múltiples veces en un entorno de pruebas.}

\subsection{Verificación}
Una vez ejecutado, puede verificar la creación correcta de las tablas consultando la vista de reporte incluida:

\begin{lstlisting}[language=SQL]
SELECT * FROM emprende_core.reporte_financiero_startups;
\end{lstlisting}

\end{document}
